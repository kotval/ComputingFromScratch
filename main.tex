\documentclass[a4paper, 12pt]{report}

%%%%%%%%%%%%
% Packages %
%%%%%%%%%%%%

\usepackage[english]{babel}
\usepackage[noheader]{packages/sleek}
\usepackage{packages/sleek-title}
\usepackage{packages/sleek-theorems}
\usepackage{packages/sleek-listings}
\usepackage{listings}
\usepackage{color}

\definecolor{dkgreen}{rgb}{0,0.6,0}
\definecolor{gray}{rgb}{0.5,0.5,0.5}
\definecolor{mauve}{rgb}{0.58,0,0.82}

\lstset{frame=tb,
  language=bash,
  aboveskip=3mm,
  belowskip=3mm,
  showstringspaces=false,
  columns=flexible,
  basicstyle={\small\ttfamily},
  numbers=none,
  numberstyle=\tiny\color{gray},
  keywordstyle=\color{blue},
  commentstyle=\color{dkgreen},
  stringstyle=\color{mauve},
  breaklines=true,
  breakatwhitespace=true,
  tabsize=3
}


%%%%%%%%%%%%%%%%
% Bibliography %
%%%%%%%%%%%%%%%%

\addbibresource{./resources/bib/references.bib}

%%%%%%%%%%
% Others %
%%%%%%%%%%


%%%%%%%%%%%%
% Document %
%%%%%%%%%%%%

\begin{document}
    \romantableofcontents

\chapter{1. Basic Unix Use}
    In How Linux Works \cite{howlinux}:
        \begin{itemize}
                \item Read Introduction
                \item Read Chapter 1. for the basics of the Linux system.
                \item Read as much of Chatper 2 as you can. You won't learn these commands by reading about them, but it helps to at least seem them all once. The advantage of using Free Software is that if you ever have an interested in how a part of your system works, it is not hidden from you. Do as much as you can from the CLI. Bash is really a programming language, but it's quite antiquated, and only useful for simple tasks. We'll come back to bash scripting later.

                \item Commands to practice and remember:
                
                \begin{lstlisting} 
        ls, cd, (cd ..), pwd, mkdir, touch, rm, (packag manager:) pacman -S, cat, vim (remember how to exit :qw, insert mode is i, ESC to change mode), man (for manuals)\end{lstlisting}
        \end{itemize}
\chapter{LISP}
\chapter{C Programming}
\begin{itemize}
        \item Do hacker rank C challenge as often as you can.
        \item In Effective C, read xxiv-xxv, Chapters 1-5.
        \item Then read Ch. 6 through Ch. 11 of Effective C. 
\end{itemize}
\chapter{Python Programming}
\begin{itemize}
        \item Basic OOP
        \item list comprehension
        \item itertools
	\item functools (you'll miss lisp fast)
        \item Numpy and pandas. Yay high level access to numerical computing
        \item Test driven development
        \item Flask, web app basics, http, networking, a tiny bit of JS.
        \end{itemize}
\end{document}
